\documentclass[a4paper]{article}

\usepackage{fullpage}
\usepackage{amsmath}

\newcommand{\Observed}{\operatorname{Observed}}
\newcommand{\Normal}{\mathcal{N}}
%% ================================================================
\title{Example problems for BAT}

%% ----------------------------------------------------------------
\begin{document}
\maketitle

This collection of random problem which are intended to guide API design for
BAT2. Following notation is used:

\begin{itemize}
\item $x \sim F(...)$ means that value $x$ is random and distributed according
  to $F$.
\item $x = y$ means that $x$ is equal to $y$ and dependence is deterministic.
\item $\operatorname{Observed}$ mean that value is observed in
  experiment. Actual value of data is not of interest and therefore omitted.
\item Any value that do not appear on the left side of equation is assumed to be
  known constant.
\end{itemize}

Please note that no distinction is made between observed value and parameter in
problem statement. Both are random quantities except former have experimental
data attached and should be treated differently during inference.


%% ----------------------------------------
\section{Simple histogram fitting}

\begin{equation}
  \begin{aligned}
    \vec\theta \sim& \,\mbox{Some fixed prior} \\
    n_i        \sim& \operatorname{Poisson}[ f_i(\vec\theta) ] \\
    &\Observed[n_i]\\
  \end{aligned}
\end{equation}

Very simple and standard problem. It's usually solved by numeric optimization of
likelihood function (which corresponds to flat prior). MCMC is only worthwhile
if likelihood have complicate shape or if one want to use MCMC to get good
initial approximation.

From API design point of view main challenge is what is elegant and performant
way to incorporate vector-valued observables.

%% ----------------------------------------
\section{Deconvolution}

\begin{equation}
  \begin{aligned}
    \alpha   \sim& \,\mbox{Improper flat prior} \\
    \vec\phi \sim& \Normal(0, \alpha\Omega) \\
    \vec{f}  \sim& \Normal(K\vec\phi, \Sigma) \\
    &\Observed[\vec{f}] \\
  \end{aligned}
\end{equation}

This is pretty standard hierarchical Bayes model. Main problem is to allow
specifying distributions for $\vec\phi$ and $\vec{f}$ differently. 
\end{document}
