\documentclass[a4paper]{article}

\usepackage[utf8]{inputenc}
\usepackage{fullpage}
% \usepackage{graphicx}
\usepackage{color}

%% ----------------------------------------------------------------
\newcommand{\FIXME}[1]{\textcolor{red}{#1}}
\newcommand{\citationneeded}{\textcolor{red}{$^{\mbox{\tiny{CITE?}}}$}}

\author{A. Khudyakov}
\title{Short and incomplete history of practical Bayesian inference}

%% ================================================================
\begin{document}
\maketitle

%% ----------------------------------------
\section{Introduction}

Bayesian inference is well known and widely used technique but its usage is very
unevenly distributed. For example in field of particle physics it's only
starting to see adoption. Its successes only became possible due to massive
amount of computational power available today. This is incomplete review of
history computational methods and software for Bayesian inference. It's not
complete but hopefully all most notable libraries/probabilistic programming
languages~(PPL) are present.

History of Bayesian inference started in XVIII century when Reverend Thomas
Bayes discovered Bayes theorem which was published after his death in 1763:

\begin{equation}
  P(A|B) = \frac{ P(B|A)P(A) }{ P(B) }
\end{equation}

If we interpret probabilities as state of knowledge we can use rule above to
update our knowledge about system. If for example $\theta$ is set of parameters
we want to estimate and $X$ is experimental data we can replace $A$ with
$\theta$ and $B$ with $X$ getting:

\begin{equation}
  p(\theta|X)
  = \frac{p(X|\theta)p(\theta)}{p(X)}
  = \frac{p(X|\theta)p(\theta)}{\int d\theta\, p(X|\theta)p(\theta)}
\end{equation}

Unfortunately integral in the denominator is not analytically tractable in most
cases. So one either limited to conjugated priors\footnote{Posterior
  distribution have same form as prior.} or have to use various
approximations. That naturally greatly limited possible uses.

Revolution came in 1990s when it was realized by Gelfand and
Smith\cite{gelfand1990sampling} that Markov chain Monte Carlo (MCMC) could be
used to sample from posterior distribution without knowing normalization factor.
Then BUGS\cite{gilks1994language} DSL\footnote{DSL~--- domain specific
  language.} was demonstrated at the 4th Bayesian Valencia meeting in April
1991. Notably according to~\cite{lunn2009bugs} it was developed independently
from Gelfand.

After that field saw a lot of development, new algorithms were devised, new
libraries and PPLs were created. Understanding of Markov chains improved too.
This review concentrates on features of software.  Much more detailed history of
use MCMC in statistics is presented in ch.~2 of ``Handbook of
MCMC''\cite{brooks2011handbook}.


%% ----------------------------------------
\section{Markov chain Monte-Carlo}

Computational heart of Bayesian inference is MCMC. Here is quick overview of
existing algorithms.

Markov chain is stochastic process where probability of transition only depends
on current state. Some of them are ergodic which means limiting probability of
being in state $X$ does not depend on starting point. It could be proven that MC
is ergodic if it's irreducible, recurrent, and aperiodic. MCMC is primarily
about constructing such chains and ensuring that they mix well. In other words
they're able to explore distribution in reasonable time.

% ----
\subsection{Metropolis-Hastings-Green algorithm}

Metropolis-Hastings\cite{metropolis1953equation}\cite{hastings1970monte}
(\FIXME{Green}\citationneeded) algorithm is widely used one. Let assume that one
want to sample from distribution $\pi(x)$.

\begin{enumerate}
\item Proposal
\item Acceptance
\item Repeat
\end{enumerate}

% ----
\subsection{Gibbs sampler}


% ----
\subsection{Hamiltonian Monte-Carlo}



%% ----------------------------------------
\section{Software}

Several different models exist. We only will concern ourselves with directed
Bayesian networks.

\FIXME{Definition, Local Markov assumption}

\FIXME{Example \& Pic}

\FIXME{Special case: simple inference, hierarchical Bayes}



% ----------------
\subsection{BUGS}

BUGS\cite{gilks1994language} is probably most influential software package for
Bayesian inference. It become first widely used PPL and led to creation of
several derivative languages. BUGS is acronym: Bayesian Inference Using Gibbs
Sampling although Metropolis-Hastings samples have been added later. Its
development started at 1989 and it was publicly demonstrated in 1991 (for more
details about history see retrospective paper\cite{lunn2009bugs}).

It's declarative external DSL. Language is very simple and allows to specify
probabilistic models concisely. It has stochastic variables, deterministic
variables and arrays. Unfortunately formal specification of language seemingly
doesn't exists so we have to use examples. Here is specification of linear
regression:

\begin{verbatim}
      model {
        for (i in 1:N) {
          y[i]  ~ dnorm(mu[i], tau)
          mu[i] <- alpha + beta * x[i]
        }
        alpha     ~ dnorm(m.alpha, p.alpha)
        beta      ~ dnorm(m.beta,  p.beta)
        log.sigma ~ dunif(a, b)
        sigma    <- exp(log.sigma)
        sigma.sq <- pow(sigma, 2)
        tau      <- 1 / sigma.sq
      }
\end{verbatim}

Here \texttt{\~} means ``distributed as'' (stochastic variables) and \texttt{<-}
denotes functional dependence (deterministic variables). Since language is
declarative order of declarations doesn't matter (except in some corner cases).
This also means that value could only be assigned once. Values \texttt{x} and
\texttt{N} are not defined in model and therefore must be supplied to
interpreter as external data.

BUGS is interpreted language. So model specification is first parsed, then DAG
is built from nodes (variables). Then it decides what is best way to sample each
stochastic variable and convert to some low-level representation which is later
used for execution.

Being interpreter BUGS is not known for outstanding performance. For example
it have to store each element of array as separate variable and update them in
turn.

Actually BUGS is whole family of PPL. Original BUGS was written in MODULA-2. It
was superseded by WinBUGS in 1997 which has powerful GUI which allowed to
specify models graphically and was much more comprehensive. As its names says
it's only available for Windows since it's written in Component Pascal and
depends on BlackBox
framework\footnote{\texttt{http://blackboxframework.org}}. In next decade
development started to diverge.

JAGS\cite{plummer2003jags} as BUGS clone written in C++ was presented in
2003. Its goals was ability to add new distributions easily, to make it possible
to use it on UNIX-like systems and possibly to vectorize operations on arrays.

Then development on OpenBUGS started in 2004. Initially it had following goals:
a) decouple user interface from core functionality, b) to allow use of BUGS from
other environments c) allow more platform independence. Second one resulted in
BRugs which allowed interactive use from R.


% ----------------
\subsection{Stan}

Stan\cite{carpenter2016stan}\cite{t2015stan} is another PPL. Just as BUGS it's
an external DSL and there're some similarities in syntax, but it's imperative
language while BUGS is declarative. It also have ``blessed'' interfaces for
python, R, and some other languages.

Unlike BUGS stan is compiled language. Stan programs are transpiled to C++
programs which in turn are compiled by g++/clang++. This allows to obtain good
performance at cost of convoluted compilation pipeline. C++ templates are used
extensively. All linear algebra is built upon eigen\citationneeded
library.

Another notable feature is use of Hamiltonian Monte-Carlo\cite{duane1987hybrid}
with No~U-Turns Sampler (NUTS)\cite{hoffman2014NUTS}. It's probably most
efficient form of MCMC for sampling from posterior with continuous
parameters. It however cannot handle discrete parameters. At this point only
stan, PyMC3 and LaplaceDemon package for R use this variant of MCMC. Another
limitation of Hamiltonian Monte-Carlo is need to know derivatives of probability
density. Stan uses automatic differentiation to calculate them.

\subsubsection{Types} %%

Now lets turn to language. It's explicitly statically typed. Primitive types are
\texttt{real} for IEEE754 doubles and \texttt{int} for 32-bit integers. Integers
are automatically promoted to reals but not vice versa. Other data types are
arrays, vectors, and matrices. Vector and matrices are distinct from arrays and
only they could be used in vector algebra routines. They're also restricted to
reals as their elements.

It's also possible to put constraints on possible values that variable could
contain. Notably arbitrary expressions could be used as constraints. So for
example:

\begin{verbatim}
  data {
    int<lower=1> N;
    real y[N];
  }
  parameters {
    real<lower=min(y), upper=max(y)> phi;
  }
\end{verbatim}

declares integer variable \texttt{N} which must be at least 1 and array of reals
\texttt{y}. Then \texttt{phi} is parameter variable which is constrained be in
range of values in \texttt{y} array.

This doesn't exhaust list of types supported by stan language. It distinguish
between row/column vectors and have specific data types for unit vectors,
covariance matrices, simplexes, etc. Reader should consult language
manual\cite{t2015stan} for details.

\subsubsection{Program structure} %%

Stan program is structured as sequence of blocks. All blocks are optional and
could be omitted. However they order is fixed and they must appear in same order
as in listing below. Scope of all variables extends to all subsequent blocks.

\begin{verbatim}
  functions {
    // ... function declarations and definitions ...
  }
  data {
    // ... declarations ...
  }
  transformed data {
    // ... declarations ... statements ...
  }
  parameters {
    // ... declarations ...
  }
  transformed parameters {
    // ... declarations ... statements ...
  }
  model {
    // ... declarations ... statements ...
  }
  generated quantities {
    // ... declarations ... statements ...
  }
\end{verbatim}

Function block is used to define user functions. Data block defines external
data for the model. Transformed data contains transformation of data into form
more convenient for the model and executed only once. Parameters declare model's
parameters. It's what is sampled or optimized. Transformed parameters allows to
defined variables in terms of data and parameters to be reused later. Model
block contain description of model and generated quantities are values which are
generated once per sample and saved for output alongside with parameters.

% ----------------
\subsection{PyMC}

PyMC\cite{patil2010pymc} is python library for performing Bayesian
inference. Its development started in 2003 as effort to make MCMC more
accessible for non-statisticians. Unlike BUGS and stan it's not a standalone
language but a library which could be easily embedded in larger python
applications.

To achieve good performance library makes used of NumPy\citationneeded and
hand-written FORTRAN routines. Thus a long as bulk of operations was vectorized
it performed better than BUGS otherwise it was comparable with it.

In user code each stochastic variable is represented as instance of
\texttt{Stochastic} class and deterministic relations between parameters are
modelled as instances of \texttt{Deterministic} class. Library provide several
methods of construction of \texttt{Stochastic}/\texttt{Deterministic}
values. There's number built-in subclasses of \texttt{Stochastic} (normal,
Poisson, etc.) and ordinary python functions could be turned into objects by use
of decorators\citationneeded. Also unobserved parameters and observed data are
modelled using same \texttt{Stochastic} values. Observed values just have
experimental data attached to them. Overall API makes heavy use of python's
dynamic features.


MCMC method employed by PyMC is mix of Gibbs sampling and
Metropolis-Hastings. Variables are sampled in turn like in Gibbs but
vector-valued use MH to update. It's possible to assign sampler to variable
manually but library tries best to select suitable one by default. It's possible
to add new sampler without modifying library \FIXME{Is description of algorithm
  correct?}

Another feature of library is pluggable ``backend''. In PyMC parlance
``backend'' stores data generated by chain. Storage in RAM, python pickles, text
files, SQLite database and HDF5 is supported on of the box.


% ----------------
\subsection{PyMC3}
PyMC3\cite{salvatier2016pymc3} is further development of PyMC. It was first
released in 2017 so quite recent development. It's current state of art for MCMC
in python and old PyMC now is of historical interest. It's rework of both API
which now allows to specify models in more compact way and internals. Notable
addition is use of NUTS\cite{hoffman2014NUTS} sampler for Hamiltonian
Monte-Carlo\citationneeded which greatly improved speed of convergence.

To boost performance it uses
Theano\cite{bergstra2010theano}\cite{arXiv1211.5590} which allows to generate
efficient machine code from python. Most notably it's used to evaluate gradients
which are needed for HMC using automatic differentiation.


% ----------------
\subsection{Infer.NET}

Infer.NET\cite{InferNET14} is C\# library which is developed at Microsoft
research. Its development started in 2008, and first non-beta release was made
in 2014. It's library as well but unlike PyMC it's statically typed.

For inference it uses Gibbs sampling so it's not very interesting from
algorithmic point of view. It also supports approximate deterministic
algorithms: expectation propagation\cite{minka2001expectation} and variational
message passing\cite{winn2005variational}.

% ----------------
\subsection{Anglican}

Anglican\cite{tolpin2016anglican} is embedded DSL for clojure and offers
interoperability with other JVM languages. 


%% ================================================================
\bibliographystyle{unsrt}
\bibliography{bib/bayes,bib/MCMC,bib/books,bib/CS}
\end{document}
